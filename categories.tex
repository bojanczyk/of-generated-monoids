
\section{Orbit-finitely generated categories}
\label{sec:orbit-finitely-generated-categories}

So far we have introduced obstructions, proved their soundness, and shown algorithms arise from them, under the assumption that obstructions are both sound and complete. It remains to prove completeness of obstructions. 


In our proof, we will move beyond monoid, and we will consider categories. The idea is that completeness will be proved using an inductive process, and the induction step might introduce some types, which will result in a split of the original set $Q$ into orbit-finitely many sets. The categories that we work are categories with atoms, which means that both the objects and morphisms are finitely supported sets with atoms. Also, the remaining structure of the category, namely composition of morphisms and the function 

\begin{definition}[Category with atoms]
    A category with atoms is a category where both the objects and morphisms are  finitely supported set with atoms, and  composition of morphisms is a finitely supported operation.
\end{definition}

We do not insist on the category being orbit-finite; in fact the entire point of this paper is to study categories where the set of morphisms is not orbit-finite. However, we will want the categories to be orbit-finitely generated, as we define below. For a category, a set of \emph{generators} is any subset of its morphisms, such that every other morphism in the category can be obtained by composing morphisms from the subset.  We will be mainly interested in categories which have an orbit-finite set of generators. Such  a category  will have an orbit-finite set of morphisms, since every object from the category has to appear in at least one generator. Therefore, sets of objects will be orbit-finite.  

All categories that we consider will be of a special kind: the objects will be some orbit-finite sets, and the morphisms will be some finitely supported functions between them. In other words, they will  be sub-categories of the category of orbit-finite sets, in which the objects are orbit-finite sets, and the morphisms are finitely supported functions between them. Here, a sub-category is obtained by deleting some objects and morphisms. Summing up, we will only be studying sub-categories of the category of orbit-finite sets which are generated by orbit-finite subsets of morphisms, as stated in the following definition.  
\newcommand{\generate}[1]{\tuple{#1}}



\begin{definition}[Orbit-finitely generated category]
    \label{def:orbit-finitely-generated-category}
    An \emph{orbit-finitely generated category} is any sub-category of the category of orbit-finite sets which is generated by an orbit-finite set of morphisms.
\end{definition}



\begin{myexample}[A monoid]
There is one object (i.e.~the category is a monoid) and the underlying set of this object is $\atoms$. The set of generators is the set of all functions which move at most two atoms, i.e.~which are the identity on all but at most two atoms. This set of generators is orbit-finite, since a generator is uniquely identified by the at most two atoms where it is not the identity, and the outputs for those atoms. These generators  all finitely-supported functions, and therefore the category is not orbit-finite, because the set of finitely supported functions of type $\atoms \to \atoms$ is not orbit-finite. 
\end{myexample}

\begin{myexample} Let us now give a category which has more than one object.
The objects are atoms. For each object $a \in \atoms$, the corresponding universe is $\atoms \setminus \set{a}$. The morphisms are all finitely supported bijections. This can be achieved by an orbit-finite set of generators, similarly to the previous example.
\end{myexample}

Let us discuss some consequences of this definition. 

\begin{lemma}
    In an orbit-finitely generated category, the set of objects is orbit-finite. Furthermore, the following maps are finitely supported: (a) the map which associates to each object its universe, and (b) the map which associates to each pair of objects the corresponding set of morphisms.
\end{lemma}
\begin{proof}
    Every object must appear in some morphism (e.g.~the identity morphism), and only the objects used in the generators can be used in the morphisms. Therefore, the set of objects is orbit-finite. 

    Let us now prove items (a) and (b). To prove this, we observe that the maps in items (a) and (b) can be defined in an equivariant way based on the generators, and therefore they are supported by whatever supports the generators.  Formally speaking, a generator can be seen as triple (source object, target object, morphism), with the last coordinate being a set of input/output pairs. The universe map from item (a) can be obtained by taking the domain of the morphism in the triple (the domain will depend only on the source object and not on the target object or morphism). A similar argument applies for the map from item (b):  taking the closure under composition is an equivariant operation, and therefore the set of all morphisms is supported by whatever supports the generators. 
\end{proof}

In all cases below, we will assume that the set of generators is equivariant, and therefore the same is true for the maps in items (a) and (b) from the above lemma. 


\begin{lemma}\label{lem:bounded-supports}
    Let $\Cc$ be an orbit-finitely generated category. Then $\Cc$ is orbit-finite if and only if it has bounded supports.  
\end{lemma}
\begin{proof}
    The left-to-right implication is immediate, since an orbit-finite set has bounded size supports of its elements (in this case, the elements are morphisms). Let us prove the right-to-left implication. 
    As per usual terminology, the hom-set for two objects is the set morphisms between them. Since the category has bounded supports, then the same is true for each hom-set. By \cite[Lemma 15]{bojanczyk_stefanski2019}, if a set of functions between two orbit-finite sets has bounded supports, then it is orbit-finite. Therefore, each hom-set is orbit-finite. There are orbit-finitely many hom-sets, and therefore the category itself is orbit-finite, since an orbit-finite union of orbit-finite sets is orbit-finite.
\end{proof}

The following lemma is called the FFT Lemma, because it is based on the Factorization Forest Theorem of Imre Simon, see~\cite[Theorem 6.1]{simon1990factorization}.
\begin{lemma}
    [FFT Lemma] Let $\Cc$ be an orbit-finitely generated category, and let $F$ be a finitely supported functor from $\Cc$ to a category which is a finite monoid (i.e.~there is one object and finitely many morphisms). If $\Cc$ is not orbit-finite, then there is a subcategory $\Dd \subseteq \Cc$ (i.e.~we delete some objects and morphisms) such that: 
    \begin{enumerate}
        \item $\Dd$ is orbit-finitely generated but not orbit-finite;
        \item all morphisms in $\Dd$ have the same image under $F$.
    \end{enumerate}
\end{lemma}
\begin{proof}
    Let $\Sigma$ be an orbit-finite set of generators for the morphisms in the category. 
    Define a \emph{path} in the category to be a finite sequence of composable generators. For each path, we define its height as follows: 
    \begin{enumerate}
        \item generators have height $1$;
        \item a composition of two paths of height at most $k$ has height at most $k+1$;
        \item if we take a finite sequence of paths of height at most $k$,  and they all have the same image under $F$, then their composition has height at most $k+1$.
    \end{enumerate}
    The factorization forest theorem states that there is some $k$ such that all paths have height at most $k$.  

    Assume that the category is not orbit-finite. By \cref{lem:bounded-supports}, the category has unbounded supports. 
    Let $\ell \in \set{1,\ldots,k}$ be the smallest number such that paths of height at most $\ell$ have  unbounded supports.  Let $\Gamma$ be the set of morphisms that can be obtained by composing paths of height strictly smaller than $\ell$. By definition of $\ell$, the set $\Gamma$ is orbit-finite. 
    If we compose two morphisms from $\Gamma$, then the result will have bounded size supports, since supports can at most be added. Therefore, we know that the only way to achieve unbounded supports for paths of height $\ell$ is to compose paths with the same image under $F$. Furthermore, by the pigeon-hole principle, this image can be chosen in finitely many ways, and therefore already a fixed choice of the image will give us unbounded supports. 
\end{proof}



