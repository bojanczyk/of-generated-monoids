
\section{Orbit-finitely generated categories}
\label{sec:orbit-finitely-generated-categories}
We will be working with categories where: (1) morphisms are finitely supported functions between orbit-finite sets, and (2) there is an orbit-finite set of generators. This is formalized in the following definition.

\begin{definition}
    The \emph{category of orbit-finite sets} is the category whose objects are orbit-finite sets, and morphisms are finitely supported functions between them.
\end{definition}

\newcommand{\generate}[1]{\tuple{#1}}

For a category $\Cc$ and a set $\Gamma$ of morphisms in this category, define the \emph{sub-category generated by $\Gamma$ in $\Cc$}, written as   ${\generate \Gamma}_{\Cc}$, to be the category where the objects are those that are used by at least one morphism from $\Gamma$, and the morphisms are all finite compositions of morphisms from $\Gamma$, plus the identity morphisms on the objects involved. We will always apply this construction in the case when the category is the category of orbit-finite sets, and in this case we will write $\generate \Gamma$ instead of ${\generate \Gamma}_{\Cc}$.


\begin{definition}[Orbit-finitely generated category]
    \label{def:orbit-finitely-generated-category}
    An \emph{orbit-finitely generated category} is any sub-category of the category of orbit-finite sets which is generated by an orbit-finite set of morphisms.
\end{definition}

We do not require that the set of morphisms is orbit-finite; in fact the topic of this paper is to understand when the set of morphisms is orbit-finite, and when it is not.

\begin{myexample}[A monoid]
There is one object (i.e.~the category is a monoid) and the underlying set of this object is $\atoms$. The set of generators is the set of all functions which move at most two atoms, i.e.~which are the identity on all but at most two atoms. This set of generators is orbit-finite, since a generator is uniquely identified by the at most two atoms where it is not the identity, and the outputs for those atoms. These generators  all finitely-supported functions, and therefore the category is not orbit-finite, because the set of finitely supported functions of type $\atoms \to \atoms$ is not orbit-finite. 
\end{myexample}

\begin{myexample} Let us now give a category which has more than one object.
The objects are atoms. For each object $a \in \atoms$, the corresponding universe is $\atoms \setminus \set{a}$. The morphisms are all finitely supported bijections. This can be achieved by an orbit-finite set of generators, similarly to the previous example.
\end{myexample}

Let us discuss some consequences of this definition. 

\begin{lemma}
    In an orbit-finitely generated category, the set of objects is orbit-finite. Furthermore, the following maps are finitely supported: (a) the map which associates to each object its universe, and (b) the map which associates to each pair of object the corresponding set of morphisms.
\end{lemma}
\begin{proof}
    Every object must appear in some morphism (e.g.~the identity morphism), and only the objects used in the generators can be used in the morphisms. Therefore, the set of objects is orbit-finite. 

    Let us now prove items (a) and (b). To prove this, we observe that the maps in items (a) and (b) can be defined in an equivariant way based on the generators, and therefore they are supported by whatever supports the generators.  Formally speaking, a generator can be seen as triple (source object, target object, morphism), with the last coordinate being a set of input/output pairs. The universe map from item (a) can be obtained by taking the domain of the morphism in the triple (the domain will depend only on the source object and not on the target object or morphism). A similar argument applies for the map from item (b):  taking the closure under composition is an equivariant operation, and therefore the set of all morphisms is supported by whatever supports the generators. 
\end{proof}

In all cases below, we will assume that the set of generators is equivariant, and therefore the same is true for the maps in items (a) and (b) from the above lemma. 


\begin{lemma}\label{lem:bounded-supports}
    Let $\Cc$ be an orbit-finitely generated category. Then $\Cc$ is orbit-finite if and only if it has bounded supports.  
\end{lemma}
\begin{proof}
    The left-to-right implication is immediate, since each orbit-finite set has bounded size supports of its elements (in this case, the elements are morphisms). Let us prove the right-to-left implication. 
    As per usual terminology, the hom-set for two objects is the set morhphisms between them. Since the category has bounded supports, then the same is true for each hom-set. By .., if a set of functions between two orbit-finite sets has bounded supports, then it is orbit-finite. Therefore, each hom-set is orbit-finite. There are orbit-finitely many hom-sets, and therefore the category itself is orbit-finite, since an orbit-finite union of orbit-finite sets is orbit-finite.
\end{proof}

The following lemma is called the FFT Lemma, because it is based on the Factorization Forest Theorem of Imre Simon. 
\begin{lemma}
    [FFT Lemma] Let $\Cc$ be an orbit-finitely generated category, and let $F$ be a finitely supported functor from $\Cc$ to a category which is a finite monoid (i.e.~there is one object and finitely many morphisms). If $\Cc$ is not orbit-finite, then there is a subcategory $\Dd \subseteq \Cc$ (i.e.~we delete some objects and morphisms) such that: 
    \begin{enumerate}
        \item $\Dd$ is orbit-finitely generated but not orbit-finite;
        \item all morphisms in $\Dd$ have the same image under $F$.
    \end{enumerate}
\end{lemma}
\begin{proof}
    Let $\Sigma$ be an orbit-finite set of generators for the morphisms in the category. 
    Define a \emph{path} in the category to be a finite sequence of composable generators. For each path, we define its height as follows: 
    \begin{enumerate}
        \item generators have height $1$;
        \item a composition of two paths of height at most $k$ has height at most $k+1$;
        \item if we take a finite sequence of paths of height at most $k$,  and they all have the same image under $F$, then their composition has height at most $k+1$.
    \end{enumerate}
    The factorization forest theorem states that there is some $k$ such that all paths have height at most $k$.  

    Assume that the category is not orbit-finite. By \cref{lem:bounded-supports}, the category has unbounded supports. 
    Let $\ell \in \set{1,\ldots,k}$ be the smallest number such that paths of height at most $\ell$ have  unbounded supports.  Let $\Gamma$ be the set of morphisms that can be obtained by composing paths of height strictly smaller than $\ell$. By definition of $\ell$, the set $\Gamma$ is orbit-finite. 
    If we compose two morphisms from $\Gamma$, then the result will have bounded size supports, since supports can at most be added. Therefore, we know that the only way to achieve unbounded supports for paths of height $\ell$ is to compose paths with the same image under $F$. Furthermore, by the pigeon-hole principle, this image can be chosen in finitely many ways, and therefore already a fixed choice of the image will give us unbounded supports. 
\end{proof}


\begin{corollary}\label{cor:orbit-infinite-subcategory-with-one-orbit}
    Let $\Cc$ be an orbit-finitely generated category that is orbit-infinite. Then there is a sub-category $\Dd \subseteq \Cc$ which is orbit-finitely generated, orbit-infinite, and which has one orbit of objects.
\end{corollary}
\begin{proof}
    Apply FFT to the functor which maps a morphism to the pair (orbit of source object, orbit of target object). This is a monoid homomorphism, and therefore there must be a sub-category that uses only one pair. This pair is necessarily on the diagonal, i.e.~the source orbit and target orbit are the same. 
\end{proof}

